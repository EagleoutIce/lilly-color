\documentclass{article}
\usepackage[T1]{fontenc}
\usepackage[utf8]{inputenc}
% tikz only for index
\usepackage{microtype,array,hyperref,pgffor,amssymb,tikz}
\usepackage[main=ngerman,english]{babel}
\usepackage[a4paper,left=1.5cm,right=1.5cm,top=2cm,bottom=2cm]{geometry}
\usepackage[legacy]{lilly-color}

\hypersetup{pdfborder=0 0 0,colorlinks=true,allcolors=Purple}

\usepackage{imakeidx}
\makeindex[title=Befehlsübersicht,columns=2,columnsep=.75cm]
\def\idxname{Befehlsübersicht}

\def\cmd#1{\texttt{\textcolor{Crimson}{\textbackslash#1}}}
\def\cmdlink#1{\phantomsection\label{cmd:#1}\cmd{#1}}
\def\cmdref#1{\hyperref[cmd:#1]{\cmd{#1}}}

\def\arg#1{\textit{\textcolor{Crimson!76!white}{#1}}}
\def\manArg#1{\texttt{\{\,\arg{#1}\,\}}}

\def\secref#1{\nameref{#1}}

\newenvironment{command}[3][v1.0]{\leavevmode\\[-\baselineskip]\hspace*{-1.75em}\index{#2?\cmdref{#2}}\cmdlink{#2}$\,$#3\hfill{}\texttt{#1}\nopagebreak\smallskip\\*}{\bigskip\par{}}

\parindent0pt
\setlength{\parskip}{.35\baselineskip plus .15\baselineskip minus .05\baselineskip}


\title{\textsf{\textcolor{AppleGreen}{\{}\textcolor{Crimson}{lilly-color}\textcolor{AppleGreen}{\}}}}
\author{Florian Sihler}
\date{\today}

\xdef\ColorList{\Allcolor}

\begin{document}

\maketitle

\begin{abstract}
    Dieses Paket liefert die Hauptfarben von \href{https://github.com/EagleoutIce/LILLY}{Lilly}, aber in einer unabhängigen Variante. Weiter liefert es Makros (wie \cmdref{RegisterColors}), die es erlauben Farben leicht zu definieren und in einer solchen Weise, dass Editoren die definierte Farbe ohne Umwege hervorheben können. Eine Übersicht findet sich in \secref{sec:overview}.
\end{abstract}

\tableofcontents

\section{Nutzermakros}

\begin{command}{RegisterColor}{\manArg{Name: Mode(Code)}}
    Definiert eine neue Farbe unter dem angegebenen Namen. Sie lässt sich dann auch mit \cmdref{Allcolor} erhalten. Beispiel:
\begin{verbatim}
\RegisterColor{Super: HTML(FFFCC9)}
\RegisterColor{Duper: RGB(150,160,170)}
\end{verbatim}
\end{command}

\begin{command}{RegisterColors}{\manArg{Color;List}}
    Erlaubt eine durch Semikolon getrennte Definition mehrerer Farben (mittels \cmdref{RegisterColor}) auf einmal.
\end{command}

\begin{command}{Allcolor}{}
    Liefert alle definierten Farben: \emph{\Allcolor{}}.
\end{command}

\section{Paketoptionen}
\label{sec:packetoptions}Das Paket kann mit den Optionen \texttt{legacy}/\allowbreak\texttt{nolegacy} geladen werden (letztere Option ist der Standard). Sie kontrolliert, ob das Paket die in \secref{sec:overview} aufgeführten, kleingeschriebenen Farben, ebenfalls geladen werden sollen.

\clearpage\section{Farbübersicht}
\label{sec:overview}%
\begin{center}
    Alle kleingeschriebenen Farben sind nur über die \texttt{legacy}-Option verfügbar.
\end{center}
\null\vfill
\begin{multicols}{2}
\begin{itemize}
    \foreach \col in \ColorList {%
        \item[\textcolor{\col}{$\blacktriangleright$}]\emph{\col} \hfill\textcolor{\col}{\fboxsep=2.5pt\colorbox{black}{$\blacksquare$}}\thinspace\textcolor{\col}{$\blacksquare$}%
    }%
\end{itemize}
\end{multicols}
\vfill\null
\clearpage\appendix
\phantomsection\addcontentsline{toc}{section}{\idxname}
\printindex
\end{document}